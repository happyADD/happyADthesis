PnP(Perspective-n-Point)是求解3D到2D点对运动的方法。它描述了当知道n个3D空间点及其投影位置
时如何估计相机的位姿。我们在已知固定的路径点和及其视觉特征时,可以利用PnP求解相机的位姿。

PnP问题有很多种求解方法,这里介绍两种方法:直接线性变换(DLT)和使用非线性优化的光束平差法(Bundle Adjustment,BA)。

\section{直接线性变换法}

现在需要解决的问题是:我们知道一组3D点的位置,以及他们在某个相机中的投影位置,求解改相机的位姿。如果把3D点看作另一个相机坐标系中的点的话
则也可以求解两个相机相对运动。

我们从方程\ref{eq:rt}中可以得到:
\begin{equation}
    s \begin{pmatrix}
        u_1 \\ v_1 \\ 1
    \end{pmatrix}
    =
    \begin{pmatrix}
        t1 & t2 & t3 & t4 \\
        t5 & t6 & t7 & t8 \\
        t9 & t10 & t11 & t12 
    \end{pmatrix}
    \begin{pmatrix}
        X \\ Y \\ Z \\ 1
    \end{pmatrix} = T P
\end{equation}

其中T的矩阵是增广矩阵$[R|t]$,它包含了旋转矩阵和平移向量。用最后一行把s消去,得到:
\[
u1 = \frac{t1X + t2Y + t3Z + t4}{t9X + t10Y + t11Z + t12},\quad v1 = \frac{t5X + t6Y + t7Z + t8}{t9X + t10Y + t11Z + t12}
\]
为了简化表示,定义T的行向量为:
\[
t1 = [t1,t2,t3,t4]^T,\quad t2 = [t5,t6,t7,t8]^T,\quad t3 = [t9,t10,t11,t12]^T
\]
于是有:
\[
t1^T P  - t3^T P u1 =0,\quad t2^T P - t3^T P v1 =0
\]

如果有n个特征点对,那么我们有2n个方程,需要求解的参数有12个,所以能求解出$[R|t]$的最少点对是6对。

\textbf{当超过6对时,我们就可以使用课上学过但是不考的十分有用的SVD分解来求解这个超定方程。}

\section{最小化重投影误差求解PnP}

考虑n个三维空间点P及其投影p,我们希望计算相机的位姿R和t,可以用矩阵T表示。假设某个空间点的坐标为$P_i = [x_i,y_i,z_i]^T$
,其在相机坐标系中的投影为$p_i = [u_i,v_i]^T$。其关系如下:
\begin{equation}
    s_i \begin{pmatrix} u_i \\ v_i \\ 1 \end{pmatrix} = K T \begin{pmatrix}
    X_i \\ Y_i \\ Z_i \\ 1
    \end{pmatrix}
\end{equation}
也就是: 
$s_i u_i = K T P_i$

但是,由于相机位姿位置以及观测点的噪声,该等式存在一个误差,我们把误差求和,构建最小二乘问题
,并对其最小化:
\begin{equation}
    T^* = \arg min  \frac{1}{2} \sum_{i=1}^{n} \left( u_i - K T P_i \right)^2
\end{equation}

当然,这个问题的求解不是我现在具备的数理基础
\footnote{需要的用到的知识:抽象代数(群论部分),矩阵论,优化理论(列文伯格——马夸尔特方法等)}
能解决。

值得庆幸的是,参考书\footnote{《视觉SLAM十四讲》,高翔等著,视觉导航必读之作}给出了解答。


