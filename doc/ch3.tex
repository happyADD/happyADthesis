PnP问题是知道3D-2D对应点求解相机位姿,而ICP问题则是知道3D-3D对应点求解相机位姿。

在使用雷达或者深度相机时,我们可以获得视角的深度信息,这样如果知道固定的3D点,
那么我们就能得到3D-3D对应点。

\section{SVD方法}
假设我们有一组匹配好的3D点:
\begin{center}
$P = {p_1, p_2,..., p_n} , \quad P' = {p'_1, p'_2,..., p'_n}$
\end{center}
我们要找到一个欧氏变换$R,t$,使得
\begin{center}
$\forall i, \quad p_i = R p'_i + t$
\end{center}
之后就可以定义第i点的误差项:

\begin{equation}
    e_i = p_i - (R p'_i + t)
\end{equation}
建立最小二乘问题,使得误差的平方和达到极小的$R,t$:
\begin{equation}
\min \limits_{R,t} \frac{1}{2} \sum_{i=1}^n ||p_i - (R p'_i + t)||_2^2
\end{equation}

下面来推导它的求解方法。首先,定义两组点的质心:
\begin{equation}
    p = \frac{1}{n} \sum_{i=1}^n p_i, \quad p' = \frac{1}{n} \sum_{i=1}^n p'_i.
\end{equation}
之后,展开并化简误差项:
\[
\begin{split}
    & \frac{1}{2} \sum_{i=1}^{n} || p_i - (R p'_i + t) ||^2 \\
    &= \frac{1}{2} \sum_{i=1}^{n} || p_i - R p'_i - t + p - R p' + p - R p' ||^2 \\
    &= \frac{1}{2} \sum_{i=1}^{n} || (p_i - p - R(p'_i - p')) + (p - R p' - t) ||^2 \\
    &= \frac{1}{2} \sum_{i=1}^{n} || p_i - p - R(p'_i - p') ||^2 + || p - R p' - t ||^2 +  2 (p - R p')^T (p_i - p - R(p'_i - p'))
\end{split}
\]
注意到交叉线项部分$(p_i - p - R(p'_i - p') )$求和为零,我们可以把优化目标函数化简为:
\begin{equation}
    \min \limits_{R,t} \frac{1}{2} \sum_{i=1}^n || p_i - p - R(p'_i - p') ||^2 + || p - R p' - t ||^2
\end{equation}
发现左边之和旋转矩阵R有关,而右边同时含有R和t,但之和质心有关。我们只要获得了R,令第二项为零就能得到t。
于是优化的过程为:
\begin{center}
    \begin{tabular}{l l}
        \hline
        步骤 & 操作 \\
        \hline
        1.计算每个点去质心坐标 & $q_i = p_i - p , q'_i = p'_i - p'$ \\
        2.建立以下优化问题计算R & $R^* = arg \min \limits_{R} \frac{1}{2} \sum_{i=1}^n ||q_i - Rq'_i||^2$ \\
        3.从R计算t & $t = p - R p'$ \\
        \hline
    \end{tabular}
\end{center}
这个过程中的优化问题R的误差项展开可得
\begin{equation}
    \frac{1}{2} \sum_{i=1}^n (q_i^T q_i + q^{'T}_i R^T R q_i - 2 q_i^T R q'_i)
\end{equation}
因为第一项和R无关,第二项$R^T R = I$,亦与R无关,所以优化问题可以简化为:
\begin{equation}
    \sum_{i=1}^n -q_i^T R q'_i = \sum_{i=1}^n - tr(R q'_i q_i^T) = - tr(R \sum_{i=1}^n q'_i q_i^T)
\end{equation}

计算这个最优问题即可解出答案。
