本文推导了PnP和ICP问题的模型建立与求解过程,其中PnP求解给出了两种方法,
实际上,根据特征点对的性质不同,还有很多求解方法,如:P3P,EPnP,UPnP等;
ICP问题也可以使用非线性优化的方法求解。

在实际工程中,我们可以使用开源库OpenCV中的solvePnP和solvePnPRansac函数求解PnP问题,
针对优化问题求解,在g2o、ceres、eigen等库的帮助下,我们可以快速地实现求解。

这里提供一个作者比赛\footnote{我们队伍地githup地址为:\url{https://github.com/nuaa-rm}}中的实践,这是通过RGB信息一个识别方形框六自由度的问题,
在尺度归一化的约束下,可以准确的算出目标的欧拉角。项目地址为:\url{https://github.com/happyADD/ENG2025.git}
这个项目基于ROS2建立,使用了海康威视的相机。

本学期是我学习的线性代数可谓是十分重要的一门课,曾经在一本专业书中看到过这样一句话:
\textbf{“工程师的一生要学三次线性代数:一次是为了考试,一次是为了研究,一次是为了实践。”}
\footnote{没记错应该是看书看到的,或许大约不是我梦里想出来的。}
这个学期我在学习《解析几何与线性代数A》的同时也学了《最优化方法》、《微分方程A》两门数学课,
同时还在完成着上面提到的比赛项目。如果这门课算是第一次,两外两门课算是第二次,自己的小项目算是第三次的话,
可以说,我用一个学期的实践完成了三次线性代数的学习!
\footnote{虽然以后肯定还要更加深入的学,但是四舍五入多活了三四十年。}
也感谢这学期的三位数学老师,是你们的有条理的一步一步推导让我知道更多。
